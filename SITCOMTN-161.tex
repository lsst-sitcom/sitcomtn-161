\documentclass[SE,lsstdraft,authoryear,toc]{lsstdoc}
\input{meta}

% Package imports go here.

% Local commands go here.

%If you want glossaries
%\input{aglossary.tex}
%\makeglossaries

\title{PSF assessment in the field of Abell 360 and HSM shear profile using ComCam data.}

% This can write metadata into the PDF.
% Update keywords and author information as necessary.
\hypersetup{
    pdftitle={PSF assessment in the field of Abell 360 and HSM shear profile using ComCam data.},
    pdfauthor={First Last},
    pdfkeywords={}
}

% Optional subtitle
% \setDocSubtitle{A subtitle}

\author{%
First Last
}

\setDocRef{SITCOMTN-161}
\setDocUpstreamLocation{\url{https://github.com/lsst-sitcom/sitcomtn-161}}

\date{\vcsDate}

% Optional: name of the document's curator
% \setDocCurator{The Curator of this Document}

\setDocAbstract{%
The Rubin ComCam on-sky campaign performed at the end of 2024 provided observations of the Abell 360 galaxy cluster; these data allow a preliminary study of cluster weak lensing analysis using Rubin Data Preview 1 (DP1) data. Among all the steps required for such analyses, accurate modeling of the PSF is paramount. This work uses several diagnostics, mostly based on the residuals between the second moments of stars and the PSF model, to characterize the accuracy of the PSF modeling in the A360 field. We find the level of the residuals to be sufficiently low not to hinder the measurement of the tangential shear profile around A360.
}

% Change history defined here.
% Order: oldest first.
% Fields: VERSION, DATE, DESCRIPTION, OWNER NAME.
% See LPM-51 for version number policy.
\setDocChangeRecord{%
  \addtohist{1}{YYYY-MM-DD}{Unreleased.}{First Last}
}


\begin{document}

% Create the title page.
\maketitle
% Frequently for a technote we do not want a title page  uncomment this to remove the title page and changelog.
% use \mkshorttitle to remove the extra pages

% ADD CONTENT HERE
% You can also use the \input command to include several content files.

\section{Introduction}
The Rubin ComCam on-sky observing campaign (ref to SITCOMTN-149) undertaken at the end-of-year 2024 covered seven fields, among which the low ecliptic latitude Rubin SV 38 7 field. This field contains the Abell 360 (A360) galaxy cluster, an intermediate mass cluster ($M_{500,c} = 4.3\times 10^{14} \rm{M}_\odot$ from ACT DR5 SZ Cluster Catalog, ref) at z=0.22, that we use as a commissioning demonstrator of cluster WL studies with data from the Vera C. Rubin observatory. This TechNote focuses on assessing the quality of the PSF modeling in the A360 field, as performed by the Rubin Science Pipeline for the Data Preview 1 (DP1) data release (ref to DP1 paper). PSF modeling was performed using the PSFex and Piff methods (Bertin et al., Jarvis et al.); the latter has been found more accurate (ref to the PSF section of DP1 paper, ref to PSTN-049 (https://pstn-019.lsst.io/PSTN-019.pdf), ref to LSST Science pipelines Papers) and has been used for the final modeling of DP1.

\section{Dataset}
The Rubin SV 38 7 field has been observed in $g$ (44 visits), $r$ (55 visits), $i$ (57 visits) and $z$ (27 visits) (ref to DP1 paper; ref to SITCOMTN-149; ref to RTN-095). No $u$ or $y$-band data were collected in that field. The $r$ or $i$-bands are generally used for weak lensing studies (ref to DES and HSC), the bluer bands being more affected by differential chromatic refraction (ref). For the DP1 analysis of A360 we will use the $i$-band, which received the most visits in DP1, to measure the shear profile around A360 and therefore focus on the $i$-band only for the purpose of this work. All the tests performed hereafter use data from the tracts and patches overlapping with  a $1^\circ \times 1^\circ$ square field centered on the BCG of Abell 360, at (ra,dec) = (37.86, 6.98) deg. 

The DP1 object table gathers all the properties of the objects (stars and galaxies) detected in the coadded images, in each band. The left panel in figure~\ref{fig:dither} shows the number of images that contributed to the coadds in the field of A360 and was obtained from the \code{i\_inputCount} information available in the object table. The dithering pattern of the observations is clearly visible.
\begin{figure}
\includegraphics[width=\textwidth]{Figures/psf_ellipticity_visits.png}
\caption{\emph{Left}: number of images used in the coadded data in the field of A360. \emph{Right}: modulus of the PSF ellipticity in the field. The dithering pattern appears clearly in both figures.\label{fig:dither}}
\end{figure}

The object table also includes the second moments of the object surface brightness $I_{xx,xy,yy}$ and second moments of the PSF model $I_{xx,xy,yy}^{\rm PSF}\,$, both measured using HSM (ref). A flag in the catalog identifies stars that have been used by Piff to build the PSF model; these are termed PSF stars. Additionally, a set of reserved stars (selected with the same criteria as PSF stars, but not used in the fit) are flagged in the catalog for the purpose of PSF testing (e.g. ref to the DES Y3 or DES Y6 PSF paper). 
With that selection, there are 1977 PSF stars and 229 reserved stars at our disposal to run the PSF diagnostics tests below. 


\section{PSF properties and diagnostics}

From the second moment matrix expressed in the (x,y) frame of the tracts, we define:
\begin{itemize}
\item The trace 
\begin{equation}
T = I_{xx} + I_{yy}
\end{equation}

\item The ellipticities e1, e2
\begin{eqnarray}
  e_1 &=& (I_{xx} - I_{yy})/T \\
  e_2 &=& 2 I_{xy} / T
\end{eqnarray}

\item The modulus of the ellipticity $e$ and the orientation $\theta$ of the major axis of the ellipse with respect to the x axis, given by
\begin{equation} 
e = \sqrt{e_1^2 + e_2^2}
\label{eq:modulus}
\end{equation}
\begin{equation}
\theta = 0.5 \times \arctan(e_2/e_1)
\label{eq:theta}
\end{equation}
\end{itemize}


From these, we compute the residuals between the measurements for the set of PSF (or reserved) stars and that of the PSF model at their locations.  Namely,
\begin{eqnarray}
\delta e_1 &=& e_1^{\rm meas} - e_1^{\rm model}, \, \delta e_2 = e_2^{\rm meas} - e_2^{\rm model}\\
\delta T &=& T^{\rm meas} - T^{\rm model} 
\end{eqnarray}

Before looking into the residuals in the next section, the right panel in Figure~\ref{fig:dither} displays the modulus of the PSF model ellipticity in the field. While the mean ellipticity across the field is ~0.07, there are several areas where the PSF ellipticity reaches significant values. [This is understood… see exchange with Pierre-François, Andrès  + TN-149. No clear correlation with the number of images in the coadd?]  


\subsection{PSF residuals - distributions and whisker plots}
To assess the performance of the PSF model, a first test consists in comparing the normalized distribution of the residuals defined above, for the PSF and reserved stars, to check for a possible overfitting of the PSF model (DES Y6 PSF paper 2025). As can be seen in Figure~\ref{fig:residual_distrib} (left and middle panels), the ellipticity residuals peak around zero and extend to $\sim 0.02$. The right panel shows the relative residuals of the trace, peaked around zero and not exceeding beyond $\sim 5 \%$. The residuals obtained from the PSF stars and reserved stars behave similarly, not indicating any overfitting of the PSF model.

\begin{figure}
\includegraphics[width=\textwidth]{Figures/residual_histograms.png}
\caption{Normalised distributions of the ellipticity residuals $\delta e_1$ (left), $\delta e_2$ (right), and of the relative residuals $\delta T/T$ (right). The distributions are shown for the PSF (blue) and reserved (orange) sets of stars. \label{fig:residual_distrib}}
\end{figure}


Figure~\ref{fig:whiskers} shows the variation of the PSF ellipticity and residuals across the field of A360. Each whisker is oriented according the direction of the ellipse major axis ($\theta$, Eq.(\ref{eq:theta})) and its length is proportional to the ellipticity modulus ($e$, Eq.(\ref{eq:modulus})); this is done for the PSF stars (top row) or reserved stars (bottom row). The left panel corresponds the measurements on the star themselves, the middle panel shows the corresponding PSF model, and the residuals are displayed in the right panel. A reference whisker is given for an ellipticity $e = 0.1$. The circle corresponds to a $0.5^\circ$  field around the cluster’s BCG, i.e., roughly the field we aim the WL measurements at. Looking at the residuals (right panel), we conclude that the PSF model perfoms satisfactorily, in particular as it is successful at correcting the coherent ellipticity patterns seen across the field.


\begin{figure}
\includegraphics[width=\textwidth]{Figures/whiskers_used.png}
\includegraphics[width=\textwidth]{Figures/whiskers_reserved.png}
\caption{Whikser plots for the PSF (top) and reserved stars (bottom) obtained for the measurements (left), PSF model (middle) and the residuals (right). The size of the whiskers is proportional to the ellipticity modulus and the orientation gives the direction of the major axis of the ellipse. \label{fig:whiskers}}
\end{figure}

\subsection{PSF residuals - tangential shear profile}

An important diagnostic for cluster weak lensing is to evaluate the contribution of the PSF residuals to the tangential shear signal. To do so, we compute the tangential component of the residuals $\delta e_t$ as 
\begin{equation}
\delta e_t = - \delta e_1  \cos(2 \phi) - \delta e_2  \sin(2 \phi),
\end{equation}
where $\phi$ is the position angle at the location of the computed residuals.

Figure~{\ref{fig:res_profile}} shows the corresponding binned radial profile as a function of the angular separation to the cluster center. The profile for PSF stars shows smaller error bars because of the larger number compared to the reserved set of the stars. The profiles are consistent with zero and have an amplitude more than an order of magnitude lower than the typical expected signal for the cluster with mass of A360 (see Section~\ref{sec:shear_profile}). 

\begin{figure}
\centering
\includegraphics[width=0.7\textwidth]{Figures/residual_tan_profile_CLMM.png}
\caption{Binned tangential profile of the ellipticity residuals for the PSF stars (blue circle), for the reserved stars (orange plus sign marker), and for both sets together (green cross).\label{fig:res_profile}}
\end{figure}


\subsection{PSF rho-statistics}
The $\rho$-statistics (Rowe 2010; Jarvis et al. 2016) are a set of two-point correlation functions involving PSF ellipticity and size residuals. They quantify spatial correlation errors in PSF modeling and contributions from PSF leakage. We used the LSST Science Pipelines \code{analysis\_tools}\footnote{\url{https://pipelines.lsst.io/v/daily/modules/lsst.analysis.tools}} implementation to compute the $\rho$-statistics, which internally relies on the TreeCorr software package (Jarvis et al. 2004) for the correlation function calculations. The definitions of the $\rho$-statistics as calculated by \code{analysis\_tools} are documented in the LSST pipelines and include those originally introduced by Rowe (2010), Jarvis et al. (2016), as well as an additional one defined in the context of cluster analysis by Melchior et al. (2015).

We used the set of reserved stars to calculate the rho statistics in the $i$-band, and display the results in FigureXX.

\section{HSM tangential and cross shear profile around A360 from color-cut selection}
\label{sec:shear_profile}
\appendix
% Include all the relevant bib files.
% https://lsst-texmf.lsst.io/lsstdoc.html#bibliographies
\section{References} \label{sec:bib}
\renewcommand{\refname}{} % Suppress default Bibliography section
\bibliography{local,lsst,lsst-dm,refs_ads,refs,books}

% Make sure lsst-texmf/bin/generateAcronyms.py is in your path
\section{Acronyms} \label{sec:acronyms}
\addtocounter{table}{-1}
\begin{longtable}{p{0.145\textwidth}p{0.8\textwidth}}\hline
\textbf{Acronym} & \textbf{Description}  \\\hline

DESC & Dark Energy Science Collaboration \\\hline
DMTN & DM Technical Note \\\hline
DP1 & Data Preview 1 \\\hline
HSC & Hyper Suprime-Cam \\\hline
HSM & Hierarchical Storage Management \\\hline
LSST & Legacy Survey of Space and Time (formerly Large Synoptic Survey Telescope) \\\hline
PSF & Point Spread Function \\\hline
RTN & Rubin Technical Note \\\hline
SE & System Engineering \\\hline
SRD & LSST Science Requirements; LPM-17 \\\hline
SV & Science Validation \\\hline
USDF & United States Data Facility \\\hline
WL & Weak gravitational Lens cosmic shear \\\hline
\end{longtable}

% If you want glossary uncomment below -- comment out the two lines above
%\printglossaries





\end{document}
