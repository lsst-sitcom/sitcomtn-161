\documentclass[SE,lsstdraft,authoryear,toc]{lsstdoc}
\input{meta}

% Package imports go here.

% Local commands go here.

%If you want glossaries
%\input{aglossary.tex}
%\makeglossaries

\title{PSF assessment in the field of Abell 360 and shapeHSM shear profile using LSSTComCam data}

% This can write metadata into the PDF.
% Update keywords and author information as necessary.
\hypersetup{
    pdftitle={PSF assessment in the field of Abell 360 and shapeHSM shear profile using LSSTComCam data},
    pdfauthor={First Last},
    pdfkeywords={}
}

% Optional subtitle
% \setDocSubtitle{A subtitle}

\author{%
C. Combet, A. Plazas Malagón, S. Fu, P. Adari, I. Dell'Antonio, A. Englert, M. Gorsuch, K. Laliotis, P.-F. Léget, E. Pedersen, A. von der Linden, Y. Zhang, et al. (TBC)
}

\setDocRef{SITCOMTN-161}
\setDocUpstreamLocation{\url{https://github.com/lsst-sitcom/sitcomtn-161}}

\date{\vcsDate}

% Optional: name of the document's curator
% \setDocCurator{The Curator of this Document}

\setDocAbstract{%
This work uses several diagnostics to assess the PSF modeling in the field of the galaxy cluster Abell 360, observed during the late 2024 ComCam campaign. We also check for possible impact on the measurement of the cluster's tangential and cross shear profiles, using the HSM measured shapes available in the object table.
}

% Change history defined here.
% Order: oldest first.
% Fields: VERSION, DATE, DESCRIPTION, OWNER NAME.
% See LPM-51 for version number policy.
\setDocChangeRecord{%
  \addtohist{1}{YYYY-MM-DD}{Unreleased.}{First Last}
}


\begin{document}

% Create the title page.
\maketitle
% Frequently for a technote we do not want a title page  uncomment this to remove the title page and changelog.
% use \mkshorttitle to remove the extra pages

% ADD CONTENT HERE
% You can also use the \input command to include several content files.

\appendix
% Include all the relevant bib files.
% https://lsst-texmf.lsst.io/lsstdoc.html#bibliographies
\section{References} \label{sec:bib}
\renewcommand{\refname}{} % Suppress default Bibliography section
\bibliography{local,lsst,lsst-dm,refs_ads,refs,books}

% Make sure lsst-texmf/bin/generateAcronyms.py is in your path
\section{Acronyms} \label{sec:acronyms}
\addtocounter{table}{-1}
\begin{longtable}{p{0.145\textwidth}p{0.8\textwidth}}\hline
\textbf{Acronym} & \textbf{Description}  \\\hline

DESC & Dark Energy Science Collaboration \\\hline
DM & Data Management \\\hline
DMTN & DM Technical Note \\\hline
DP1 & Data Preview 1 \\\hline
DRP & Data Release Processing \\\hline
HSC & Hyper Suprime-Cam \\\hline
HSM & Hierarchical Storage Management \\\hline
LSST & Legacy Survey of Space and Time (formerly Large Synoptic Survey Telescope) \\\hline
LSSTComCam & Rubin Commissioning Camera \\\hline
PSF & Point Spread Function \\\hline
RTN & Rubin Technical Note \\\hline
SE & System Engineering \\\hline
SNR & Signal to Noise Ratio \\\hline
SV & Science Validation \\\hline
TBC & To Be Confirmed \\\hline
USDF & United States Data Facility \\\hline
WL & Weak gravitational Lens cosmic shear \\\hline
\end{longtable}

% If you want glossary uncomment below -- comment out the two lines above
%\printglossaries





\end{document}
